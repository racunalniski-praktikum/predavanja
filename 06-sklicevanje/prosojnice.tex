\documentclass[]{beamer}
\usepackage{../pomozno/prosojnice}
\setbeamertemplate{theorems}[numbered]
{\theoremstyle{plain}
  \newtheorem{izrek}{Izrek}
  \newtheorem{posledica}[izrek]{Posledica}
}

% Definicija okoli za definicije in vaje
{\theoremstyle{definition}
  \newtheorem{definicija}[izrek]{Definicija}
  \newtheorem{vaja}[izrek]{Vaja}
}

\begin{document}
\begin{frame}[fragile]
  \begin{center}
    \Huge\textbf{Sklicevanje}
  \end{center}
\end{frame}

\setcounter{izrek}{5}
\begin{frame}[fragile]
  \begin{izrek}[Pitagorov izrek]
    \label{izr:pitagora}
    V pravokotnem trikotniku s katetama \(a\) in \(b\) ter hipotenuzo \(c\) velja
    \[
      a^2 + b^2 = c^2.
    \]
  \end{izrek}
  Izrek~\ref{izr:pitagora} je eden najstarejših izrekov v matematiki.
  \begin{slaboenv}
    \begin{minted}{latex}
\textbf{Izrek~6 (Pitagorov izrek)} \\
\emph{V pravokotnem trikotniku s katetama \(a\) in \(b\)
ter hipotenuzo \(c\) velja
\[ a^2 + b^2 = c^2. \]}
Izrek~6 je eden najstarejših izrekov v matematiki.
    \end{minted}
  \end{slaboenv}
\end{frame}

\setcounter{izrek}{5}
\begin{frame}[fragile]
  \begin{izrek}[Pitagorov izrek]
    % \label{izr:pitagora}
    V pravokotnem trikotniku s katetama \(a\) in \(b\) ter hipotenuzo \(c\) velja
    \[
      a^2 + b^2 = c^2.
    \]
  \end{izrek}
  Izrek~\ref{izr:pitagora} je eden najstarejših izrekov
  v matematiki.
  \begin{dobroenv}
    \begin{minted}{latex}
\begin{izrek}[Pitagorov izrek]
  \label{izr:pitagora}
  V pravokotnem trikotniku s katetama \(a\) in \(b\)
  ter hipotenuzo \(c\) velja
  \[ a^2 + b^2 = c^2. \]
\end{izrek}
Izrek~\ref{izr:pitagora} je eden najstarejših izrekov
v matematiki.
    \end{minted}
  \end{dobroenv}
\end{frame}

\begin{frame}[fragile]
  \begin{minted}{latex}
{\theoremstyle{plain}
\newtheorem{izrek}{Izrek}[section]
\newtheorem{posledica}[izrek]{Posledica}
}

{\theoremstyle{definition}
\newtheorem{definicija}[izrek]{Definicija}
\newtheorem{vaja}[izrek]{Vaja}
}
  \end{minted}
\end{frame}

\begin{frame}[fragile]
  \begin{minted}{bibtex}
@book{vidav08,
  author    = {Ivan Vidav},
  title     = {Višja matematika I (12.~ponatis)},
  publisher = {DMFA--Založništvo},
  year      = 2008
}
@book{rudin87,
  author    = {Walter Rudin},
  title     = {Real and Complex Analysis},
  publisher = {McGraw-Hill},
  year      = 1987
}
@unpublished{zapiski,
  author = {Sašo Strle},
  title  = {Analiza 1 za študente Finančne matematike},
  note   = {zapiski predavanj},
  year   = 2020
}
  \end{minted}
\end{frame}

\begin{frame}[fragile]
  \begin{minted}{latex}
Lorem ipsum dolor sit amet, consectetur adipiscing elit.
Nulla semper, ligula eget porttitor lobortis, lacus ante
posuere magna, eget dapibus lacus odio sit amet urna.
Curabitur eget tincidunt lacus~\cite{vidav08}, a mollis
lectus. Donec non est a eros gravida venenatis nec quis
elit. Sed vulputate~\cite{analiza1}, neque eget
pellentesque egestas, eros diam porttitor est, non
lobortis velit turpis vitae velit.

\bibliographystyle{plain}
\bibliography{literatura}
  \end{minted}
\end{frame}

\end{document}
