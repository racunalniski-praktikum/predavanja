\documentclass{beamer}
\usepackage{../pomozno/prosojnice}
\DeclareMathOperator{\sinus}{sinus}
\setminted{
    fontsize=\scriptsize
}

\title{Najpogostejše napake v \LaTeX-u}
\date{}

\begin{document}
\frame{\titlepage}

\begin{frame}[fragile]{Ne vklapljajte nepotrebnih paketov}
    \begin{columns}[t]
        \begin{column}{0.5\textwidth}
            \begin{dobroenv}
            \begin{minted}{latex}
\documentclass[11pt]{article}

\usepackage[utf8]{inputenc}
\usepackage[T1]{fontenc}

\usepackage[slovene]{babel}
\usepackage{lmodern}
\usepackage{amsmath}
\usepackage{amsfonts}
\usepackage{amsthm}
\usepackage{mathtools}
            \end{minted}   
            \end{dobroenv}
        \end{column}
        \begin{column}{0.5\textwidth}
            \begin{slaboenv}
            \begin{minted}{latex}
\documentclass[11pt]{article}

\usepackage[utf8]{inputenc}
% Ostale možnosti za kodiranje so:
% Kodirna tabela za Windows
% \usepackage[cp1250]{inputenc}
% Kodirna tabela za Linux
% \usepackage[latin2]{inputenc}
\usepackage[T1]{fontenc}

\usepackage[slovene]{babel}
\usepackage{lmodern}
\usepackage{amsmath}
\usepackage{amsfonts}
\usepackage{amsthm}
\usepackage{mathtools}
\usepackage{graphicx}
% \usepackage{tikz}
            \end{minted} 
        \end{slaboenv}       
        \end{column}
    \end{columns}
\end{frame}

\begin{frame}[fragile]{Če uporabljate \texttt{align}, uporabite tudi \texttt{\&}}
    \begin{dobroenv}
    \begin{minted}{latex}
    \begin{align*}
        a_{11} x_1 + a_{12} x_2 + \dots + a_{1n} x_n &= b_1 \\
        a_{21} x_1 + a_{22} x_2 + \dots + a_{2n} x_n &= b_2 \\
                                                     &\vdotswithin{=} \\
        a_{n1} x_1 + a_{n2} x_2 + \dots + a_{nn} x_n &= b_n,
    \end{align*}
    \end{minted}

    \begin{align*}
        a_{11} x_1 + a_{12} x_2 + \dots + a_{1n} x_n &= b_1 \\
        a_{21} x_1 + a_{22} x_2 + \dots + a_{2n} x_n &= b_2 \\
                                                     &\vdotswithin{=} \\
        a_{n1} x_1 + a_{n2} x_2 + \dots + a_{nn} x_n &= b_n,
    \end{align*}
    \end{dobroenv}
\end{frame}

\begin{frame}[fragile]{Če uporabljate \texttt{align}, uporabite tudi \texttt{\&}}
    \begin{slaboenv}
    \begin{minted}{latex}
        \begin{align*}
            a_{11} x_1 + a_{12} x_2 + \dots + a_{1n} x_n = b_1 \\
            a_{21} x_1 + a_{22} x_2 + \dots + a_{2n} x_n = b_2 \\
                                                         \vdotswithin{=} \\
            a_{n1} x_1 + a_{n2} x_2 + \dots + a_{nn} x_n = b_n,
        \end{align*}
        \end{minted}

    \begin{align*}
        a_{11} x_1 + a_{12} x_2 + \dots + a_{1n} x_n = b_1 \\
        a_{21} x_1 + a_{22} x_2 + \dots + a_{2n} x_n = b_2 \\
                                                     \vdotswithin{=} \\
        a_{n1} x_1 + a_{n2} x_2 + \dots + a_{nn} x_n = b_n,
    \end{align*}
    \end{slaboenv}
\end{frame}

\begin{frame}[fragile]{Če uporabljate \texttt{align}, uporabite tudi \texttt{\&}}
    \begin{slaboenv}
    \begin{minted}{latex}
        \begin{align*}
            a_{11} x_1 + a_{12} x_2 + \dots + a_{1n} x_n = b_1 \\
            a_{21} x_1 + a_{22} x_2 + \dots + a_{2n} x_n = b_1 + b_2 \\
                                                         \vdotswithin{=} \\
            a_{n1} x_1 + a_{n2} x_2 + \dots + a_{nn} x_n = b_1 + b_2 + \dots + b_n,
        \end{align*}
        \end{minted}

    \begin{align*}
        a_{11} x_1 + a_{12} x_2 + \dots + a_{1n} x_n = b_1 \\
        a_{21} x_1 + a_{22} x_2 + \dots + a_{2n} x_n = b_1 + b_2 \\
                                                     \vdotswithin{=} \\
        a_{n1} x_1 + a_{n2} x_2 + \dots + a_{nn} x_n = b_1 + b_2 + \dots + b_n,
    \end{align*}
    \end{slaboenv}
\end{frame}

\begin{frame}[fragile]{Če nočete oštevilčenja, uporabite \texttt{align*}}
    \begin{dobroenv}
    \begin{minted}{latex}
        \begin{align*}
            a_{11} x_1 + a_{12} x_2 + \dots + a_{1n} x_n = b_1 \\
            a_{21} x_1 + a_{22} x_2 + \dots + a_{2n} x_n = b_2 \\
                                                         \vdotswithin{=} \\
            a_{n1} x_1 + a_{n2} x_2 + \dots + a_{nn} x_n = b_n,
        \end{align*}
        \end{minted}

    \begin{align*}
        a_{11} x_1 + a_{12} x_2 + \dots + a_{1n} x_n = b_1 \\
        a_{21} x_1 + a_{22} x_2 + \dots + a_{2n} x_n = b_2 \\
                                                     \vdotswithin{=} \\
        a_{n1} x_1 + a_{n2} x_2 + \dots + a_{nn} x_n = b_n,
    \end{align*}
    \end{dobroenv}
\end{frame}

\begin{frame}[fragile]{Če nočete oštevilčenja, uporabite \texttt{align*}}
    \begin{slaboenv}
    \begin{minted}{latex}
        \begin{align}
            a_{11} x_1 + a_{12} x_2 + \dots + a_{1n} x_n = b_1 \\
            a_{21} x_1 + a_{22} x_2 + \dots + a_{2n} x_n = b_2 \\
                                                         \vdotswithin{=} \\
            a_{n1} x_1 + a_{n2} x_2 + \dots + a_{nn} x_n = b_n,
        \end{align}
        \end{minted}

    \begin{align}
        a_{11} x_1 + a_{12} x_2 + \dots + a_{1n} x_n = b_1 \\
        a_{21} x_1 + a_{22} x_2 + \dots + a_{2n} x_n = b_2 \\
                                                     \vdotswithin{=} \\
        a_{n1} x_1 + a_{n2} x_2 + \dots + a_{nn} x_n = b_n,
    \end{align}
    \end{slaboenv}
\end{frame}

\begin{frame}[fragile]{Če nočete oštevilčenja, uporabite \texttt{align*}}
    \begin{slaboenv}
    \begin{minted}{latex}
    \begin{align}
        a_{11} x_1 + a_{12} x_2 + \dots + a_{1n} x_n = b_1 \nonumber\\
        a_{21} x_1 + a_{22} x_2 + \dots + a_{2n} x_n = b_2 \nonumber\\
                                                        \vdotswithin{=}  \nonumber\\
        a_{n1} x_1 + a_{n2} x_2 + \dots + a_{nn} x_n = b_n, \nonumber
    \end{align}
    \end{minted}

    \begin{align}
        a_{11} x_1 + a_{12} x_2 + \dots + a_{1n} x_n = b_1 \nonumber\\
        a_{21} x_1 + a_{22} x_2 + \dots + a_{2n} x_n = b_2 \nonumber\\
                                                     \vdotswithin{=}  \nonumber\\
        a_{n1} x_1 + a_{n2} x_2 + \dots + a_{nn} x_n = b_n, \nonumber
    \end{align}
    \end{slaboenv}
\end{frame}

\begin{frame}[fragile]{Matematične ukaze definirajte brez \texttt{\$}}
    \begin{slaboenv}
    \begin{minted}{latex}
    \newcommand{\Rn}{$\mathbb{R}^n$}
    Kjer je matrika s koeficienti iz \Rn, $b \in$ \Rn je stolpec skalarjev,
    $x \in$ \Rn pa stolpec neznank.
    \end{minted}
    \newcommand{\Rn}{$\mathbb{R}^n$}
    Kjer je matrika s koeficienti iz \Rn, $b \in$ \Rn je stolpec skalarjev,
    $x \in$ \Rn pa stolpec neznank.
    \end{slaboenv}

    \begin{dobroenv}
    \begin{minted}{latex}
    \newcommand{\Rn}{\mathbb{R}^n}
    Kjer je matrika s koeficienti iz $\Rn$, $b \in \Rn$ je stolpec skalarjev,
    $x \in \Rn$ pa stolpec neznank.
    \end{minted}
    \newcommand{\Rn}{\mathbb{R}^n}
    Kjer je matrika s koeficienti iz $\Rn$, $b \in \Rn$ je stolpec skalarjev,
    $x \in \Rn$ pa stolpec neznank.
    \end{dobroenv}
\end{frame}

\begin{frame}[fragile]{Uporabite lahko tudi \texttt{\textbackslash ensuremath}}
    \begin{sprejemljivoenv}
    \begin{minted}{latex}
    \newcommand{\Rn}{\ensuremath{\mathbb{R}^n}}
    Kjer je matrika s koeficienti iz \Rn, $b \in \Rn$ je stolpec skalarjev,
    $x \in \Rn$ pa stolpec neznank.
    \end{minted}
    \newcommand{\Rn}{\ensuremath{\mathbb{R}^n}}
    Kjer je matrika s koeficienti iz \Rn, $b \in \Rn$ je stolpec skalarjev, $x \in \Rn$ pa stolpec neznank.
    \end{sprejemljivoenv}

    \begin{dobroenv}[Bolje]
    \begin{minted}{latex}
    \newcommand{\Rn}{\mathbb{R}^n}
    Kjer je matrika s koeficienti iz $\Rn$, $b \in \Rn$ je stolpec skalarjev,
    $x \in \Rn$ pa stolpec neznank.
    \end{minted}
    \newcommand{\Rn}{\mathbb{R}^n}
    Kjer je matrika s koeficienti iz $\Rn$, $b \in \Rn$ je stolpec skalarjev, $x \in \Rn$ pa stolpec neznank.
    \end{dobroenv}
\end{frame}


\begin{frame}[fragile]{Uporabiti morate ustrezne ukaze}
    \begin{slaboenv}
    \begin{minted}{latex}
    \Sigma_{i=1}^n a_i^2 = 42
    \end{minted}
    $$\Sigma_{i=1}^n a_i^2 = 42$$
    \end{slaboenv}

    \begin{dobroenv}
    \begin{minted}{latex}
    \sum_{i=1}^n a_i^2 = 42
    \end{minted}
    \begin{center}
    $$2 \left(\left(\left(\sum_{i=1}^n a_i^2 = 42\right)\right)\right)$$
    \end{center}
    \end{dobroenv}
\end{frame}

\begin{frame}[fragile]{Uporabiti morate ustrezne ukaze}
    \begin{slaboenv}
    \begin{minted}{latex}
    x \epsilon \mathbb{R}
    \end{minted}
    $$x \epsilon \mathbb{R}$$
    \end{slaboenv}

    \begin{dobroenv}
    \begin{minted}{latex}
    x \in \mathbb{R}
    \end{minted}
    $$x \in \mathbb{R}$$
    \end{dobroenv}
\end{frame}


\begin{frame}[fragile]{Uporabiti morate ustrezne ukaze}
    \begin{slaboenv}
    \begin{minted}{latex}
    A \cap B = \phi
    \end{minted}
    $$A \cap B = \phi$$
    \end{slaboenv}

    \begin{dobroenv}
    \begin{minted}{latex}
    A \cap B = \emptyset
    \end{minted}
    $$A \cap B = \emptyset$$
    \end{dobroenv}
\end{frame}

\begin{frame}[fragile]{Uporabiti morate ustrezne ukaze}
    \begin{slaboenv}
    \begin{minted}{latex}
    $2 < x , y > + 3 < y, x >$
    \end{minted}
    $10 \left< x , y \right> + 3 \left< y, x \right>$
    \end{slaboenv}

    \begin{dobroenv}
    \begin{minted}{latex}
    $2 \langle x , y \rangle + 3 \langle y, x \rangle$
    \end{minted}
    $2 \left< x , y \right> + 3 \langle y, x \rangle$
    \end{dobroenv}
\end{frame}


\begin{frame}[fragile]{Za matematične operacije uporabimo ustrezne ukaze}
\begin{columns}[t]
\begin{column}{0.45\textwidth}
\begin{slaboenv}
\begin{minted}{latex}
sin x + sin (x + y)
\end{minted}
$$sin x + sin (x + y)$$
\end{slaboenv}
\begin{slaboenv}
\begin{minted}{latex}
sin~x + sin~(x + y)
\end{minted}
$$sin~x + sin~(x + y)$$
\end{slaboenv}
\begin{dobroenv}
\begin{minted}{latex}
\sin x + \sin (x + y)
\end{minted}
$$\sin x + \sin (x + y)$$
\end{dobroenv}
\end{column}
\begin{column}{0.55\textwidth}
\begin{slaboenv}
\begin{minted}{latex}
\mathsf{sin}~x + \mathsf{sin}~(x + y)
\end{minted}
$$\mathsf{sin}~x + \mathsf{sin}~(x + y)$$
\end{slaboenv}
\begin{slaboenv}
\begin{minted}{latex}
\text{sin}~x + \text{sin}~(x + y)
\end{minted}
$$\text{sin}~x + \text{sin}~(x + y)$$
\end{slaboenv}
\begin{dobroenv}
\begin{minted}{latex}
\DeclareMathOperator{\sinus}{sinus}
\sinus x + \sinus (x + y)
\end{minted}
$$\sinus x + \sinus (x + y)$$
\end{dobroenv}
\end{column}
\end{columns}
\end{frame}

\begin{frame}[fragile]{Novih odstavkov se ne dela z \texttt{\textbackslash\textbackslash}}
    \begin{dobroenv}
    \begin{minted}{latex}
    Lorem ipsum dolor sit amet, consectetur adipiscing elit.
    Ut id viverra ligula. Phasellus vehicula lorem vitae luctus
    dignissim.

    Sed ac justo commodo, fringilla urna ac, efficitur leo.
    Praesent dui odio, accumsan ac sapien nec, interdum volutpat
    est. 
    \end{minted}

    Lorem ipsum dolor sit amet, consectetur adipiscing elit.
    Ut id viverra ligula. Phasellus vehicula lorem vitae luctus
    dignissim.

    Sed ac justo commodo, fringilla urna ac, efficitur leo.
    Praesent dui odio, accumsan ac sapien nec, interdum volutpat
    est.
    \end{dobroenv}
\end{frame}

\begin{frame}[fragile]{Novih odstavkov se ne dela z \texttt{\textbackslash\textbackslash}}
    \begin{sprejemljivoenv}
    \begin{minted}{latex}
    Lorem ipsum dolor sit amet, consectetur adipiscing elit.
    Ut id viverra ligula. Phasellus vehicula lorem vitae luctus
    dignissim. \par
    Sed ac justo commodo, fringilla urna ac, efficitur leo.
    Praesent dui odio, accumsan ac sapien nec, interdum volutpat
    est. 
    \end{minted}

    Lorem ipsum dolor sit amet, consectetur adipiscing elit.
    Ut id viverra ligula. Phasellus vehicula lorem vitae luctus
    dignissim. \par
    Sed ac justo commodo, fringilla urna ac, efficitur leo.
    Praesent dui odio, accumsan ac sapien nec, interdum volutpat
    est.
    \end{sprejemljivoenv}
\end{frame}

\begin{frame}[fragile]{Novih odstavkov se ne dela z \texttt{\textbackslash\textbackslash}}
    \begin{slaboenv}
    \begin{minted}{latex}
    Lorem ipsum dolor sit amet, consectetur adipiscing elit.
    Ut id viverra ligula. Phasellus vehicula lorem vitae luctus
    dignissim. \\
    Sed ac justo commodo, fringilla urna ac, efficitur leo.
    Praesent dui odio, accumsan ac sapien nec, interdum volutpat
    est. 
    \end{minted}

    Lorem ipsum dolor sit amet, consectetur adipiscing elit.
    Ut id viverra ligula. Phasellus vehicula lorem vitae luctus
    dignissim. \\
    Sed ac justo commodo, fringilla urna ac, efficitur leo.
    Praesent dui odio, accumsan ac sapien nec, interdum volutpat
    est.
    \end{slaboenv}
\end{frame}

\begin{frame}[fragile]{Novih odstavkov se ne dela z \texttt{\textbackslash\textbackslash}}
    \begin{slaboenv}
    \begin{minted}{latex}
    Lorem ipsum dolor sit amet, consectetur adipiscing elit.
    Ut id viverra ligula. Phasellus vehicula lorem vitae luctus
    dignissim. \\[1ex]
    Sed ac justo commodo, fringilla urna ac, efficitur leo.
    Praesent dui odio, accumsan ac sapien nec, interdum volutpat
    est. 
    \end{minted}

    Lorem ipsum dolor sit amet, consectetur adipiscing elit.
    Ut id viverra ligula. Phasellus vehicula lorem vitae luctus
    dignissim. \\[1ex]
    Sed ac justo commodo, fringilla urna ac, efficitur leo.
    Praesent dui odio, accumsan ac sapien nec, interdum volutpat
    est.
    \end{slaboenv}
\end{frame}

\begin{frame}[fragile]{Za sredinsko poravnane izraze uporabljamo prikazni način}
    \begin{slaboenv}
    \begin{minted}{latex}
    \begin{center}
        \( e^{i \pi} + 1 = 0 \)
    \end{center}
    \end{minted}
    \begin{center}
        \( e^{i \pi} + 1 = 0 \)
    \end{center}
    \end{slaboenv}

    \begin{dobroenv}
    \begin{minted}{latex}
    \[ % oz. \begin{equation*}
        e^{i \pi} + 1 = 0
    \] % oz. \end{equation*}
    \end{minted}
    \[ e^{i \pi} + 1 = 0 \]
    \end{dobroenv}
\end{frame}

\begin{frame}[fragile]{Za sredinsko poravnane izraze uporabljamo prikazni način}
    \begin{slaboenv}
    \begin{minted}{latex}
    \begin{center}
        \( x_{1, 2} = \frac{-b \pm \sqrt{b^2 - 4 ac}}{2a} \)
    \end{center}
    \end{minted}
    \begin{center}
        \( x_{1, 2} = \frac{-b \pm \sqrt{b^2 - 4 ac}}{2a} \)
    \end{center}
    \end{slaboenv}

    \begin{dobroenv}
    \begin{minted}{latex}
    \[ % oz. \begin{equation*}
        x_{1, 2} = \frac{-b \pm \sqrt{b^2 - 4 ac}}{2a}
    \] % oz. \end{equation*}
    \end{minted}
    \[ x_{1, 2} = \frac{-b \pm \sqrt{b^2 - 4 ac}}{2a} \]
    \end{dobroenv}
\end{frame}

\end{document}