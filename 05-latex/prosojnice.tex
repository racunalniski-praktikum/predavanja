\documentclass[aspectratio=169]{beamer}
\usepackage{../pomozno/prosojnice}

\begin{document}
\begin{frame}[fragile]
  \begin{center}
    \Huge\textbf \LaTeX
  \end{center}
\end{frame}

\begin{frame}
  \vspace{1em}
  \begin{center}
    {\Huge Donald E. Knuth}
    \bigskip

    \includegraphics[height=0.65\textheight,keepaspectratio]{knuth.jpg}
    \smallskip

    dobitnik Turingove nagrade (1974)
  \end{center}
\end{frame}

\usebackgroundtemplate{%
  \includegraphics[width=\paperwidth,height=\paperheight]{knuth-katja.jpg}%
}
\begin{frame}
\end{frame}
\usebackgroundtemplate{}

\usebackgroundtemplate{%
  \includegraphics[width=\paperwidth,height=\paperheight]{taocp.jpg}%
}
\begin{frame}
\end{frame}
\usebackgroundtemplate{}

\begin{frame}
  \vspace{1em}
  \begin{center}
    {\Huge
      \begin{itemize}
        \item Vol.~1 (1968)
        \item Vol.~2 (1969)
        \item Vol.~3 (1973)
        \item<2> Vol.~2, druga izdaja (1976)
        \item Vol.~4A (2011)
        \item Vol.~4B (2022)
      \end{itemize}
    }
  \end{center}
\end{frame}

\begin{frame}
  \vspace{1em}
  \begin{center}
    {\Huge \TeX\ (1978)}
  \end{center}
\end{frame}

\begin{frame}
  \vspace{1em}
  \begin{center}
    {\Huge Leslie Lamport}
    \bigskip

    \includegraphics[height=0.65\textheight,keepaspectratio]{lamport.jpg}
    \smallskip

    dobitnik Turingove nagrade (2013)
  \end{center}
\end{frame}

\begin{frame}
  \vspace{1em}
  \begin{center}
    {\Huge \LaTeX\ (1984)}
  \end{center}
\end{frame}

\begin{frame}[fragile]{Markdown}

\begin{minted}{markdown}
# Naslov

Besedilo je lahko **poudarjeno** ter
vsebuje [povezave](https://www.fmf.uni-lj.si/),
`programsko kodo` in sezname:
- prva točka
- druga točka
- tretja točka
\end{minted}

\end{frame}

\begin{frame}[fragile]{HTML}

\begin{minted}{html}
<h1>Naslov</h1>

<p>Besedilo je lahko <strong>poudarjeno</strong> ter
vsebuje <a href="https://www.fmf.uni-lj.si/">povezave</a>,
<code>programsko kodo</code> in sezname:</p>
<ul>
  <li>prva točka</li>
  <li>druga točka</li>
  <li>tretja točka</li>
</ul>
\end{minted}

\end{frame}

\begin{frame}[fragile]{\LaTeX}

\begin{minted}{latex}
\section{Naslov}

Besedilo je lahko \textbf{poudarjeno} ter
vsebuje \href{https://www.fmf.uni-lj.si/}{povezave},
\texttt{programsko kodo} in sezname:

\begin{itemize}
  \item prva točka
  \item druga točka
  \item tretja točka
\end{itemize}
\end{minted}

\end{frame}

\begin{frame}[fragile]{Ukazi in okolja}

  \begin{block}{Ukazi}
    \begin{itemize}
      \item \mintinline{latex}|\ukaz|
      \item \mintinline{latex}|\ukaz{argument}|
      \item \mintinline{latex}|\ukaz[neobvezen]{argument}|
      \item \mintinline{latex}|\ukaz[neobvezen]{argument1}{argument2}|
    \end{itemize}
  \end{block}

  \begin{block}{Okolja}
    \begin{minted}{latex}
  \begin{okolje}
    ...
  \end{okolje}
    \end{minted}
  \end{block}
\end{frame}

\begin{frame}[fragile]{Struktura dokumenta}

\begin{minted}{latex}
\documentclass[11pt,a4paper]{article}
% preambula
\usepackage[utf8]{inputenc}
\usepackage[T1]{fontenc}
\usepackage[slovene]{babel}

\title{Naslov dokumenta}
\author{Avtor dokumenta}
\date{\today}

\begin{document}
\maketitle
...vsebina...
\end{document}
\end{minted}
\end{frame}

\begin{frame}[fragile]{Matematika}
  \centering

  $$e^{i \pi} + 1 = 0$$
  \mintinline{latex}|e^{i \pi} + 1 = 0|

  $$\lim_{x \to 0} \frac{\sin x}{x} = 1$$
  \mintinline{latex}|\lim_{x \to 0} \frac{\sin x}{x} = 1|

  $$\int_{-\infty}^\infty e^{-x^2} \, dx = \sqrt{\pi}$$
  \mintinline{latex}|\int_{-\infty}^\infty e^{-x^2}\,dx = \sqrt{\pi}|
\end{frame}

\begin{frame}[fragile]{Vrstični in prikazni način}
  Naj bodo \(a, b, c \in \mathbb{R}\). Tedaj je
  \[
    x_{1, 2} = \frac{-b \pm \sqrt{b^2 - 4ac}}{2a}
  \]
  rešitev enačbe \(ax^2 + bx + c = 0\).

  \begin{minted}{latex}
Naj bodo \(a, b, c \in \mathbb{R}\). Tedaj je
\[
  x_{1, 2} = \frac{-b \pm \sqrt{b^2 - 4ac}}{2a}
\]
rešitev enačbe \(ax^2 + bx + c = 0\).
  \end{minted}
\end{frame}

\begin{frame}[fragile]{Vrstični in prikazni način}
  Naj bodo $a, b, c \in \mathbb{R}$. Tedaj je
  $$
  x_{1, 2} = \frac{-b \pm \sqrt{b^2 - 4ac}}{2a}
  $$
  rešitev enačbe $ax^2 + bx + c = 0$.

  \begin{minted}{latex}
Naj bodo $a, b, c \in \mathbb{R}$. Tedaj je
$$
  x_{1, 2} = \frac{-b \pm \sqrt{b^2 - 4ac}}{2a}
$$
rešitev enačbe $ax^2 + bx + c = 0$.
  \end{minted}
\end{frame}

\begin{frame}[fragile]{Matematična okolja}
  \begin{align*}
	a x^2 + b x + c &= 0 \\
	\left(x + \frac{b}{2a}\right)^2 &= \frac{b^2-4ac}{4a^2} \\
	x_{1, 2} &= \frac{-b \pm \sqrt{b^2-4ac}}{2a}
  \end{align*}

  \begin{minted}{latex}
\begin{align*}
a_{11} x_1 + a_{12} x_2 + \cdots + a_{1n} x_n &= b_1 \\
a_{21} x_1 + a_{22} x_2 + \cdots + a_{2n} x_n &= b_2 \\
&\vdots \\
a_{m1} x_1 + a_{m2} x_2 + \cdots + a_{mn} x_n &= b_m
\end{align*}
  \end{minted}
\end{frame}

\begin{frame}[fragile]
  \begin{center}
    \Huge MathJaX
  \end{center}
\end{frame}

\end{document}
